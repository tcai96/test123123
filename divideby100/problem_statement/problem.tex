

\problemname{Divide by 100...}

Dividing two numbers and computing the decimals is an extremely difficult task. Luckily, dividing a number by a ``special'' number is very easy (at least for us humans)!
 
We will define the set of ``special'' numbers $S=\{10^K\}$ for all non-negative integers $K$, i.e. $\{1,10,100,\ldots\}$.

Given a large numbers $N$ and a ``special'' large number $M$, what does the decimal representation of $$\frac{N}{M}$$ look like?

\section*{Input}

The first line of input contains 2 integers $N$, $M$, where $1\leq N, M\leq 10^{10^6}$, and $M\in S$.

\section*{Output}

Print the \emph{exact} decimal preresentation of $\frac{N}{M}$, i.e. every digit, \emph{without} trailing zeroes; if the quotient is less than $1$, print one leading zero (see sample input).
