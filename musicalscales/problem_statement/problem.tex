\problemname{Musical Scales}

The following are musical notes in ``increasing order'':
\begin{center}
    A, A\#, B, C, C\#, D, D\#, E, F, F\#, G, G\#
\end{center}
The difference between consecutive notes is a {\em semitone}, and the sequence
wraps around so the note that is one semitone above G\# is A.
The difference between a {\em tone} is the same as two semitones. So the note that is one tone above B is C\#.
The note that is one tone above G is A.

We do not worry
about flats such as Cb nor do we worry about adding a \# sign to B and E
in this problem (they are aliases for notes that are already listed).

A major scale is defined by a note (such as A or C\#) and all other notes
following that one in an arithmetic progression:
\begin{center}
    {\em tone}, {\em tone}, {\em semitone}, {\em tone}, {\em tone}, {\em tone}, {\em semitone}
\end{center}
The starting note appears in the name of the scale.

For example, the scale A\#-major consists of the following notes:
\begin{center}
    A\#, C, D, D\#, F, G, A, A\#
\end{center}
(by convention, the first note is repeated at the end of the sequence)

Finally, in this problem a song is just a sequence of notes. Your job is to
identify all major scales such that the song uses only notes in that scale.

\section*{Input}

The first line of input is an integer $1\leq n \leq 100$ denoting the number of notes played in a song.
The second line consists of a sequence of notes, separated by spaces.

\section*{Output}

Output consists of a single line that lists all scales the song may be played in.
Consecutive scales should be separated by a single space and the scales must appear
in lexicographic order. If the song may not fit in any one of these scales, simply output
a line containing the text \texttt{none}.
